\documentclass{beamer}
\usetheme{Singapore}

%\usepackage{pstricks,pst-node,pst-tree}
\usepackage{amssymb,latexsym}
\usepackage{graphicx}
\usepackage{fancyvrb}
\usepackage{relsize}
\usepackage{multicol}


\newcommand{\bi}{\begin{itemize}}
\newcommand{\ii}{\item}
\newcommand{\ei}{\end{itemize}}
\newcommand{\Show}[1]{\psshadowbox{#1}}


\newcommand{\grf}[2]{\centerline{\includegraphics[width=#1\textwidth]{#2}}}
\newcommand{\tw}{\textwidth}
\newcommand{\bc}{\begin{columns}}
\newcommand{\ec}{\end{columns}}
\newcommand{\cc}[1]{\column{#1\textwidth}}

\newcommand{\bfr}[1]{\begin{frame}[fragile]\frametitle{{ #1 }}}
\newcommand{\efr}{\end{frame}}

\newcommand{\cola}[1]{\begin{columns}\begin{column}{#1\textwidth}}
\newcommand{\colb}[1]{\end{column}\begin{column}{#1\textwidth}}
\newcommand{\colc}{\end{column}\end{columns}}

\title{Little Book of Semaphores, Chapter 4}
\author{Geoffrey Matthews\\
\small Western Washington University}

\RecustomVerbatimEnvironment{Verbatim}{Verbatim}
{frame=single,fontsize=\relsize{-1}}

\begin{document}
\maketitle

\bfr{Producers and Consumers}
\begin{multicols}{2}
\begin{Verbatim}[label=Producer i]
event[i] = waitForEvent()
buffer.add(event[i])
\end{Verbatim}
\begin{Verbatim}[label=Consumer j]
event[j] = buffer.get()
event[j].process()
\end{Verbatim}
\end{multicols}
\bi 
\ii Threads must have exclusive
access to the buffer.  No adding and getting at the same time.  
\ii If
a consumer thread arrives when the buffer is empty, it blocks until a
producer adds an item.
\ii {\tt waitForEvent} and {\tt process} can happen simultaneously,
but not buffer access.
\ii {\tt event} is a local variable for each thread---not shared.
\ii We could use an array, as above, with all producers and consumers
having different indices.
\ei

\end{frame}

\bfr{Producers and Consumers Hint}
\begin{Verbatim}
mutex = Semaphore(1)
items = Semaphore(0)
local event
\end{Verbatim}
\bi
\ii Local events can be handled several ways:
\bi
\ii Each thread has its own run-time stack.  (We use this in scheme
and python, where threads are functions.)
\ii Threads could be objects, with local private variables.
\ii Threads can use unique IDs as indices into an array.
\ei
\ei
\end{frame}


\bfr{Producer-consumer solution}
\begin{multicols}{2}
\begin{Verbatim}[label=Producer]
event = waitForEvent()
mutex.wait()
  buffer.add(event)
  items.signal()
mutex.signal()
\end{Verbatim}
\begin{Verbatim}[label=Consumer]
items.wait()
mutex.wait()
  event = buffer.get()
mutex.signal()
event.process()
\end{Verbatim}
\end{multicols}
\pause
\bi
\ii Could the {\tt items.signal()} be taken out of the mutex?
\ii What would be the advantage?
\ei


\end{frame}

\bfr{Producer-consumer solution (slight improvement)}
\begin{multicols}{2}
\begin{Verbatim}[label=Producer]
event = waitForEvent()
mutex.wait()
  buffer.add(event)
mutex.signal()
items.signal()
\end{Verbatim}
\begin{Verbatim}[label=Consumer]
items.wait()
mutex.wait()
  event = buffer.get()
mutex.signal()
event.process()
\end{Verbatim}
\end{multicols}
\pause
\bi
\ii {\tt items} could at times not accurately reflect the actual
number of waiting consumers.  
\ei
\end{frame}

\bfr{Producer-consumer solution (broken)}
\begin{multicols}{2}
\begin{Verbatim}[label=Producer]
event = waitForEvent()
mutex.wait()
  buffer.add(event)
mutex.signal()
items.signal()
\end{Verbatim}
\begin{Verbatim}[label=Consumer]
mutex.wait()
  items.wait()
  event = buffer.get()
mutex.signal()
event.process()
\end{Verbatim}
\end{multicols}
\bi
\ii Why is this broken?
\pause
\ii Don't wait for a semaphore after grabbing a mutex!
\ei
\end{frame}

\bfr{Producer-consumer with finite buffer}
\begin{Verbatim}[label=Broken finite buffer solution]
if items >= bufferSize:
  block()
\end{Verbatim}
\bi
\ii {\tt items} is a semaphore, we can't check its size
\ii Even if we could, we could be interrupted between checking and
blocking.
\pause
\ii What to do?
\ei
\end{frame}

\bfr{Finite buffer producer-consumer hint}
\begin{Verbatim}
mutex = Semaphore(1)
items = Semaphore(0)
spaces = Semaphore(bufferSize)
\end{Verbatim}
\end{frame}

\bfr{Finite buffer producer-consumer solution}
\begin{multicols}{2}
\begin{Verbatim}[label=Producer]
event = waitForEvent()

spaces.wait()
mutex.wait()
  buffer.add(event)
mutex.signal()
items.signal()
\end{Verbatim}
\begin{Verbatim}[label=Consumer]
items.wait()
mutex.wait()
  event = buffer.get()
mutex.signal()
spaces.signal()

event.process()
\end{Verbatim}
\end{multicols}
\end{frame}

\bfr{Readers-writers problem}
\bi
\ii Suppose a number of process all access the same data.
\ii Any number of readers can be in the critical section
simultaneously.
\ii Writers must have exclusive access to the critical section.
\ii Ideas?
\ei
\end{frame}

\bfr{Readers-writers hint}
\begin{Verbatim}
readers = 0
mutex = Semaphore(1)
roomEmpty = Semaphore(1)
\end{Verbatim}

\bi
\ii ``wait'' means ``wait for the condition to be true''
\ii ``signal'' means ``signal that the condition is true''
\ei
\end{frame}

\bfr{Readers-writers solution}
\begin{multicols}{2}
\begin{Verbatim}[label=Writers]
roomEmpty.wait()
# critical section for writer
roomEmpty.signal()












\end{Verbatim}
\begin{Verbatim}[label=Readers]
mutex.wait()
  readers += 1
  if readers == 1:
    # first in locks:
    roomEmpty.wait()
mutex.signal()

# critical section for reader

mutex.wait()
  readers -= 1
  if readers == 0:
    # last out unlocks
    roomEmpty.signal()
mutex.signal()
\end{Verbatim}
\end{multicols}

\bi
\ii A {\bf Lightswitch}
\ei

\end{frame}

\bfr{A lightswitch object}
\begin{minipage}{0.5\textwidth}
\begin{Verbatim}
class Lightswitch:
  def __init__(self):
    self.counter = 0
    self.mutex = Semaphore(1)

  def lock(self, semaphore):
    self.mutex.wait()
      self.counter += 1
      if self.counter == 1:
        semaphore.wait()
    self.mutex.signal()

  def unlock(self, semaphore):
    self.mutex.wait()
      self.counter -= 1
      if self.counter == 0:
        semaphore.signal()
        self.mutex.signal()
\end{Verbatim}
\end{minipage}\begin{minipage}{0.5\textwidth}
\begin{Verbatim}[label=Initialization]
readswitch = Lightswitch()
roomEmpty = Semaphore(1)
\end{Verbatim}
\vfill
\begin{Verbatim}[label=Readers]
readswitch.lock(roomEmpty)
# critical section
readwitch.unlock(roomEmpty)
\end{Verbatim}
\vfill
\begin{Verbatim}[label=Writers]
roomEmpty.wait()
# critical section for writer
roomEmpty.signal()
\end{Verbatim}
\vfill
\end{minipage}
\end{frame}        

\bfr{Starvation}

\bi
\ii No deadlock in the above readers-writers solution.
\ii However, it is possible for a writer to {\bf starve}.
\ii While a writer is blocked, readers can come and go, and the writer
never progresses.
\ii (In the buffer problem, readers eventually empty the buffer, but
we can imagine readers who simply examine the buffer without removing
an item.)
\pause
\ii Puzzle:  extend the solution so that when a writer arrives, the
existing readers can finish, but no additional readers may enter.
\ei


\end{frame}        

\bfr{No-starve readers-writers hint}
\begin{Verbatim}
readSwitch = Lightswitch()
roomEmpty = Semaphore(1)
turnstile = Semaphore(1)
\end{Verbatim}
\pause
\bi
\ii {\tt turnstile} is a turnstile for readers and a mutex for writers
\ei

\end{frame}        

\bfr{No-starve readers-writers solution}
\begin{multicols}{2}
\begin{Verbatim}[label=Writers]
turnstile.wait()
  roomEmpty.wait()
  # critical section
turnstile.signal()

roomEmpty.signal()
\end{Verbatim}
\begin{Verbatim}[label=Readers]
turnstile.wait()
turnstile.signal()

readSwitch.lock(roomEmpty)
  # critical section
readSwitch.unlock(roomEmpty)
\end{Verbatim}
\end{multicols}
\bi
\ii {\tt turnstile} is a turnstile for readers and a mutex for writers
\pause
\ii It is now possible for {\em readers} to starve!
\ei

\end{frame}        

\bfr{Priority Scheduling}
\bi
\ii Some schedulers allow priority scheduling.
\ii Puzzle:  Write a solution to readers-writers that gives priority
to writers.  In other words, once a writer arrives, no readers are
allowed in the critical section until {\em all} writers have left the
system. 
\ei
\end{frame}        

\bfr{Writer-priority readers-writers hint}
\begin{Verbatim}
readSwitch = Lightswitch()
writeSwitch = Lightswitch()
mutex = Semaphore(1)
noReaders = Semaphore(1)
noWriters = Semaphore(1)
\end{Verbatim}
\end{frame}

\bfr{Writer-priority readers-writers solution}
\begin{multicols}{2}
\begin{Verbatim}[label=Writers]
writeSwith.lock(noReaders)
  noWriters.wait()

    # critical section

  noWriters.signal()
writeSwitch.unlock(noReaders)
\end{Verbatim}
\begin{Verbatim}[label=Readers]
noReaders.wait()
  readSwitch.lock(noWriters)
noReaders.signal()

# critical section

readSwitch.unlock(noWriters)
\end{Verbatim}
\end{multicols}
\bi

\ii Writers in critical section hold {\em both} {\tt noReaders}
and {\tt noWriters}.
\ii {\tt writeSwitch} allows writers to queue on {\tt noWriters}, but
keeps {\tt noReaders} locked
\ii The last writer signals {\tt noReaders}
\pause
\ii Readers in critical section
 hold {\tt noWriters} but don't hold {\tt noReaders}, so a writer can
 lock {\tt noReaders}
\ii The last reader signals {\tt noWriters} so writers can go
\ei
\end{frame}

\bfr{Thread starvation}
\bi
\ii We just addressed {\bf categorical starvation}:   one category of
threads makes another category starve.
\ii {\bf Thread starvation} is the more general possibility of a
thread waiting indefinitely while other threads proceed.
\pause
\ii Part of the problem is the responsibility of the scheduler.  If a
thread is never scheduled, it is starved.
\pause
\ii Some schedulers use algorithms that guarantee bounded waiting.
\ei
\end{frame}

\bfr{Thread starvation}
\bi
\ii If we don't want to assume too much about the scheduler, can we
assume: 
\ii
{\bf Property 1:}  if there is only one thread that is ready to run, the
scheduler has to let it run.
\ii This would be sufficient for the boundary problem.
\pause
\ii In general we need a stronger assumption:
\ii 
{\bf Property 2:}  if a thread is ready to run, then the time it waits
until it runs is bounded.
\pause
\ii We use this assumption in all our work.
\ii Some schedulers in the real world do not guarantee this strictly.
\ei
\end{frame}

\bfr{Semaphore starvation}
\bi
\ii When one thread signals a semaphore, which waiting thread is woken
up?
\pause
\ii {\bf Property 3:}  if there are threads waiting on a semaphore
when a thread executes {\tt signal}, then one of the waiting threads
has to be woken.
\pause
\ii Prevents a thread from signalling a semaphore, racing
around a loop, waiting on the same semaphore, and catching its own
signal! 
\pause
\ii However, if A, B, and C are using a mutex in a loop, A and B could
race around and around, starving C.
\pause
\ii A semaphore with Property 3 is called a {\bf weak semaphore}.
\ei
\end{frame}

\bfr{Semaphore starvation}
\bi
\ii {\bf Property 4:}  if a thread is waiting at a semaphore, then the
number of threads that will be woken before it is bounded.
\pause
\ii FIFO queues satisfy this property.
\pause
\ii A semaphore with Property 4 is called a {\bf strong semaphore}.
\pause
\ii Dijkstra (inventor of semaphores) conjectured in 1965 that it was
impossible to solve the mutex problem without starvation with weak
semaphores. 
\pause
\ii Morris showed you could do it in 1979. (Solution in book.)
\ei
\end{frame}
\end{document}
