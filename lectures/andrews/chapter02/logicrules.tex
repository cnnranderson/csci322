\documentclass[12pt]{article}
\sloppy
\usepackage{multicol}
\usepackage{natded}

\setlength{\parindent}{0in}

\newcommand{\lang}{\ensuremath{\langle}}
\newcommand{\rang}{\ensuremath{\rangle}}
\newcommand{\eq}{&\equiv&}
\newcommand{\imp}{\Rightarrow}
\newcommand{\bi}{\begin{itemize}}
\newcommand{\ii}{\item}
\newcommand{\ei}{\end{itemize}}
\newcommand{\stm}[3]{\ensuremath{\{#1\}}\ \mbox{\tt #2}\ \ensuremath{\{#3\}}}
\newcommand{\la}{\ensuremath{\langle}}
\newcommand{\ra}{\ensuremath{\rangle}}

\begin{document}


\centerline{\bf Logic: axioms}

\begin{multicols}{2}
\begin{eqnarray*}
A\land(B\lor C) \eq  (A\land B) \lor (A \land C)\\
A\lor(B\land C) \eq  (A\lor B) \land (A \lor C)
\end{eqnarray*}
\begin{eqnarray*}
\neg \neg A \eq A\\
A\lor A \eq A\\
A\lor \neg A \eq true\\
A\lor true \eq true\\
A\lor false \eq A\\
A\land true \eq A\\
A\land false \eq false\\
A\land A \eq A\\
A\land \neg A \eq false\\
A\imp true \eq true\\
A\imp false \eq \neg A\\
true\imp A \eq A\\
false \imp A \eq true\\
A\imp A \eq true\\
A\imp B \eq \neg A \lor B\\
A\imp B \eq \neg B \imp \neg A\\
\neg(A\imp B) \eq A \land \neg B\\
A\land(A\lor B) \eq A\\
A\lor(A\land B) \eq A\\
A\land(\neg A \lor B) \eq A \land B\\
A \lor (\neg A \land B) \eq A \lor B\\
\neg(A\land B) \eq \neg A \lor \neg B\\
\neg(A\lor B) \eq \neg A \land \neg B
\end{eqnarray*}

\end{multicols}



\centerline{\bf Logic: inference rules}
\begin{multicols}{2}
\bi
\ii Modus Ponens
\[
\frac{A\imp B, A}{B}
\]
\ii Modus Tollens
\[
\frac{A\imp B, \neg B}{\neg A}
\]
\ii Conjunction
\[
\frac{A,B}{A\land B}
\]
\ii Simplification
\[
\frac{A\land B}{A}
\]
\ii Addition
\[
\frac{A}{A\lor B}
\]
\ii Disjunctive syllogism
\[
\frac{A\lor B, \neg A}{B}
\]
\ii Hypothetical syllogism
\[
\frac{A\imp B, B\imp C}{A\imp C}
\]
\ii Constructive dilemma
\[
\frac{A\lor B, A\imp C, B\imp D}{C\lor D}
\]
\ii Destructive dilemma
\[
\frac{\neg C\lor \neg D, A\imp C, B\imp D}{\neg A \lor \neg B}
\]
\ei
\end{multicols}

\newpage
\centerline{\bf Axiomatic Semantics}

{\bf Assignment Axiom}
\[
\stm{P_{x\leftarrow t}}{x = t}{P}
\]

{\bf Inference Rules}
\bi
\ii Composition
\[
\frac{\stm{P}{S1}{R}, \stm{R}{S2}{Q}}{\stm{P}{S1;S2}{Q}}
\]

\begin{multicols}{2}
\ii Consequence
\[
\frac{P\imp R, \stm{R}{S}{Q}}{\stm{P}{S}{Q}}
\]

\[
\frac{\stm{P}{S}{T}, T\imp Q}{\stm{P}{S}{Q}}
\]
\end{multicols}


\ii If-then
\[
\frac{\stm{P\land C}{S}{Q}, P\land\neg C\imp Q}{\stm{P}{if C then
    S}{Q}}
\]
\ii If-then-else
\[
\frac{\stm{P\land C}{S1}{Q}, \stm{P\land\neg C}{S2}{Q}}{\stm{P}{if C then
    S1 else S2}{Q}}
\]

\ii While
\[
\frac{\stm{P\land C}{S}{P}}{\stm{P}{while C do S}{P\land\neg C}}
\]

\ei

\vfill

\centerline{\bf Sematics of Concurrent Execution}
\bi
\ii Await rule
\[
\frac{\stm{P\land B}{S}{Q}}{\stm{P}{\la await (B) S;\ra}{Q}}
\]
\ii Co rule
\[
\frac{\stm{P_i}{Si}{Q_i}\mbox{\ are interference free}}
     {\stm{P_1\land ... \land P_n}{co S1; // ... // Sn; oc}{Q_1\land ... \land Q_n}}
\]
\ii One process {\bf interferes} with another if it executes an
assignment that invalidates an assertion in the other process. 
\ei




\end{document}
