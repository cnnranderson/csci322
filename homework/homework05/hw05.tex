\documentclass{article}
\pagestyle{empty}
\sloppy
\usepackage[margin=1.5in]{geometry}
\title{Message Passing}
\author{Homework \#5, CSCI 322, Winter 2015\\
Geoffrey Matthews}

\begin{document}
\maketitle
\begin{description}
\item[Description:] \ 

This is based on Exercise 7.9 from the text.  
Implement your solutions in {\bf Racket} using only channels for
synchronization.  You may choose to use either synchronous or
asynchronous channels.

Two kinds of processes, {\tt A}'s and {\tt B}'s, enter a room.  An
{\tt A} process cannot leave until it meets two {\tt B} processes, and
a {\tt B} process cannot leave until it meets one {\tt A} process.
Each kind of process leaves the room---without meeting any other
processes---once it has met the required number of other processes.

{(a)}  
Develop a server process to implement this synchronization.  Show the
interface of {\tt A} and {\tt B} processes to the server.

{(b)} 
Modify your answer to (a) so that the first of the two {\tt B}
processes that meets an {\tt A} process does not leave the room until
after the {\tt A} process meets a second {\tt B} process.

\item[Turn in:]\ 

(a) A single racket file demonstrating your solution to (a)

(b) A single racket file demonstrating your solution to (b)

(c) A writeup, explaining your strategy for both solutions.

\item[Due date:] \ 

Wednesday, March 11, midnight.


\end{description}
\end{document}