\documentclass{article}
\pagestyle{empty}
\sloppy
\usepackage[margin=1.5in]{geometry}
\title{Philosophers!}
\author{Homework \#4, CSCI 322, Winter 2015\\
Geoffrey Matthews}

\begin{document}
\maketitle
\begin{description}
\item[Description:] This is based on Exercise 4.17 from the text.  

Solve
  the dining philosophers problem by focusing on the state of the
  philosophers rather than the forks.  In particular, let {\tt
    eating[n]} be a boolean array;  {\tt eating[i]} is true if {\tt
    Philosopher[i]} is eating and is false otherwise.

\hspace{-1em}{(a)}  
Specify a global invariant, then develop a coarse-grained
solution, and finally develop a fine-grained solution that uses
semaphores for synchronization.  Your solution should be
deadlock-free, but an individual philosopher might starve.  Use the
technique of passing the baton.  Describe how your fine-grained
solution matches the semantics of the coarse-grained solution.

Implement your fine-grained solution in either {\bf Racket}, {\bf
  Python} or {\bf Pthreads}.  Be sure to use only semaphores for
synchronization and mutual exclusion.  Name the semaphores well.
Insert your invariant at relevant places as an assertion, {\em i.e.},
the evaluation of a boolean expression or expressions which, if false,
halts the program with an error message.

\hspace{-1em}{(b)} Modify your answer to (a) to avoid starvation.  In particular, if
a philosopher wants to eat, eventually he or she gets to eat.
Describe your strategy in clear language.  If it lends itself to an
invariant (not all strategies do), state that clearly.

Develop both coarse-grained and fine-grained solutions, and implement
the fine-grained solution in your chosen language.

  
\item[Writeup:] There should be three files submitted:
\begin{itemize}
\item The writeup, which includes both parts, justifications, {\em
    etc.}  It should look something like the examples in the
  textbooks.  You can use Andrews-like pseudocode here.
\item A runnable (or compilable on unix) solution to part (a).
\item A runnable (or compilable on unix) solution to part (b).
\end{itemize}

\item[Due date:] Wednesday, February 25, midnight.



\end{description}
\end{document}