\documentclass{article}
\pagestyle{empty}
\sloppy
\usepackage[margin=1.5in]{geometry}
\title{Planets!}
\author{Homework \#2, CSCI 322, Winter 2015\\
Geoffrey Matthews}

\begin{document}
\maketitle
\begin{description}
\item[Description:] This is a continuation of homework \#1.  Fix any
  errors in your version of {\tt planets02.rkt} (see me if you need
  help) and then create two new versions of this program:
\begin{enumerate}
\item A program called {\tt planets03.rkt} that uses a new thread for
  each planet, and also uses the synchronization methods in {\em The
    Little Book of Semaphors}, Listing 3.10, to assure that all planet
  forces are updated before any planet moves, and that all planets
  move before any update their forces.
\item A program called {\tt planets04.rkt} that behaves just like {\tt
    planets03.rkt} except that the thread that refreshes the view {\em
    does not execute} while the planets are updating their positions.
  It can update the view any time the planets are calculating their
  forces, but not when they are changing their positions.  Make sure
  you use {\em only basic semaphores} ({\tt semaphore-post} and {\tt
    semaphore-wait}) to accomplish this task.  

  Give meaningful names to all your semaphores, and explain in the
  comments in the file how your program assures that refresh never
  happens while the planets are moving.  If you use an example from
  {\em The Little Book}, reference it and discuss how it works in our
  example.
  
\end{enumerate}
\item[Due date:] Friday, February 6, midnight.
  50\% of this assignment's grade points for each stage.


\end{description}
\end{document}