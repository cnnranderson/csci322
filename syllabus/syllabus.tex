\documentclass{article}
\usepackage[margin=0.75in]{geometry}
\usepackage{alltt}
\usepackage{multicol}
\usepackage{hyperref}
\begin{document}

\centerline{\large \bf Syllabus}

\centerline{\bf CSCI 322, Principles of Concurrent Programming, Winter, 2013}

\begin{itemize}

\item
{\bf Instructor:} Geoffrey Matthews, x3797, {\tt  geoffrey dot matthews at wwu dot edu}, CF 469
\item
{\bf Office hours:} MTWF 11:00
\item
{\bf Texts and Readings:} 
%\begin{multicols}{2}
\begin{itemize}
\item {\em The Little Book of Semaphores}, Allen B. Downey, 
\url{http://www.greenteapress.com/semaphores/}
\item {\em Foundations of Multithreaded, Parallel, and Distributed Programming}, Gregory Andrews, Addison-Wesley, 2000.
\item Other readings as assigned.
\end{itemize}
%\end{multicols}
\item {\bf Webpage:}  \url{wwu.instructure.com}  homework assignments, grades
\item {\bf Repository:} \url{https://github.com/geofmatthews/csci322}\  handouts, lectures, code
\item
{\bf Lectures:} Two sections:
\begin{tabular}{rrr}
CRN 12129 & AW402 & MTWF 10:00\\
CRN 13459 & AH14 & MTWF 12:00\\
\end{tabular}
\item
{\bf Content:}
This course will cover key concepts, both abstract and concrete, in
concurrent programming.  Topics covered include:
\begin{multicols}{2}
\begin{itemize}
\item Shared variable and message passing paradigms
\item Mutual exclusion and synchronization techniques
\item Semaphores 
\item Condition variables
\item Monitors 
\item Rendezvous 
\item Concurrent programming facilities in high-level languages
\item Paradigms for process interaction:  bag of tasks, heartbeat,
  pipeline, {\em etc.}
\end{itemize}
\end{multicols}
\item
{\bf Course goals:}
On completion of this course, students will demonstrate: 
\begin{multicols}{2}
  \begin{enumerate}
  \item Thorough understanding of problems and techniques in
    concurrent programming.
\item Thorough understanding of algorithms, techniques and programming
  language support for mutual exclusion, synchronization and
  communication between processes and threads.
\item Thorough understanding of the purpose, implementation and use of
  semaphores, monitors, and rendezvous in concurrent programming.
\item The effective use of programming language facilities and
  libraries in the design and implementation of multi-threaded and
  multi-process applications.

  \end{enumerate}
\end{multicols}
\item {\bf Software:} Programming in Scheme, Python, and C, using
  various libraries such as Pthreads, MPI, and OpenMP.  We will also
  examine case studies of concurrency solutions in other languages,
  such as Ada and Java.

\item {\bf Exams:}   One
  midterm and one final.  You may bring two pages of notes to use
  during the exams.

\item {\bf Quizzes:} Except for the first and last weeks of class and
  exam days, we will have weekly quizzes on Fridays.  The lowest two
  quiz scores will be dropped

\item {\bf Reading:} All students are expected to do the online
  reading assigned throughout the quarter in order to be prepared for
  the weekly quizzes and the exams.

\item {\bf Homework:}  Homework assignments will be passed out
  regularly through the quarter, involving a mix of theory (math) and
  programming.   Homework will be due at the start of
  class on the due date.  Late work is accepted at a penalty
  of 25\% per each fraction of 24 hours late.

\item {\bf Attendance:}  I will periodically take attendance in
  class.  Perfect attendance is worth 5 points of extra credit.  There
  are no other extra credit opportunities in this class.

\item {\bf Grading:} \\
$0\leq F < 60 \leq D < 70 \leq C < 80 \leq B < 90 \leq A$\hfill
\begin{tabular}{|l|l|l|l|}\hline
Homework & Quizzes & Midterm & Final\\\hline
40\% & 20\% & 15\% & 25\%\\\hline
\end{tabular}

\item {\bf Academic dishonesty:} Academic dishonesty policy and
  procedure is discussed in the University Catalog, Appendix D.  All
  students should read this section of the catalog.  Academic
  dishonesty consists of misrepresentation by deception or other
  fraudulent means.  In computer science courses this frequently takes
  the form of copying another's program, either a fellow student's
  program, or copying one from the web.  Due diligence should be
  exercised in the labs at all times, since both copying and letting
  someone else copy your program are equally culpable.  Do not walk
  away from your computer in the lab without logging out or locking
  the screen.  Do not share files, even if it is just to ``show them
  something.''  Describe it in words, or talk to them in person, never
  share code.

\item {\bf Collaboration:} Collaboration with your fellow students is
  a good way to learn.  Feel free to share ideas, solve problems, and
  discuss your programs with other students.  However, collaboration
  is {\em not} copying.  All code should be original.  Remember the
  {\bf Simpson's Rule:} after discussing homework with another
  student, each of you must destroy all written notes, pictures,
  files, {\em etc.} that you shared.  After that, you must watch a
  rerun of {\em the Simpson's}, or do something else unrelated, for
  half an hour.  Then you can take the knowledge you gained from
  another student and put it to work, since it is now not copying, but
  learning.  You have made it your own.

\item
{\bf Approximate Schedule:} The following schedule may be adjusted
radically depending on interests and problems as they occur.

\begin{tabular}{|l|l|l|}\hline
Book & Begins\\\hline
TLBOS  1-4 & January 7 \\
FOMPADP 1-2 & January 19 \\
FOMPADP 3 & January 26\\
FOMPADP 4 & February 2\\\hline
\multicolumn{2}{|c|}{Midterm Friday February 6} \\\hline
FOMPADP 5 & February 9\\
FOMPADP 7 & February 16\\
FOMPADP 8 & February 23\\
FOMPADP 9 & March 2\\
Review &  March 9 \\\hline
{CRN 12129 10:00 lecture}&{Final Monday March 16 8:00am}\\\hline
{CRN 13459 12:00 lecture}&{Final Monday March 16 1:00pm}\\\hline
\end{tabular}
\end{itemize}

\end{document}
